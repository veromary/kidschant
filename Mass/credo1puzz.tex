\documentclass[12pt,a4paper]{article}
\usepackage{cwpuzzle}
\usepackage{libertine}
\usepackage{gregoriotex}

\title{Credo part 1}
\author{Veronica Brandt}

\begin{document}

\autocompilegabc


\maketitle

\section{Credo - first part}

\begin{description}
\item[Credo] I believe.  Credemus = we believe, credes = you believe, credent = they believe, credet = he/she/it believes.  From this we get words such as credit, credible, incredible.
\item[in] in
\item[unum] one (the ending is to match Deum).  Unus is masculine, una is feminine, unum is neuter and they all mean one.  Count in Latin: unus, duo, tres, quatuor, quinque, sex, septem, octo, novem, decem!
\item[Deum] God - Deus is the subject form of God.  Deum is the object form of God.  So in this case, the verb is Believe, the subject is I and the object is God.  I believe in God.
\item[Patrem] Father - Pater is the subject form of Father.  The `m' on the end tells us that this is the object.  In English we see the same effect with he and him.  He is the subject, the object is him.
\item[omnipotentem] almighty, all powerful (potent = power)
\item[factorem] maker.  A factory is where they make things.
\item[caeli] of heaven.  Caelum is the subject form of heaven.  Caeli could mean heavens or of heaven.
\item[et] and
\item[terrae] of earth.  Terra is the subject form of earth.  Terrae could mean earths or of earth.
\item[visibilium] visible
\item[omnium] all things.  The ending -um is to match visibilium.  Later we will also see omnia and omnes.
\item[et invisibilium] and invisible.
\end{description}

\section{The Chant}

This is Credo 1 - the default setting for the Creed.  There are six settings in the Kyriale.  

They all can start with the same tune, except for Credo 3.  Credo 3 is the most modern setting, so some say it is easier to learn, but we'll start by learning the number 1 setting.  It's simpler that way.

Here are the first few phrases:

\includescore{credo1a}

You can see how the notes generally rise and fall --- look at the little hill shape at \emph{visibilium omnium}.

There are a couple of high bits with a little `b' symbol --- look at \emph{Patrem} and the \emph{et} in \emph{et invisibilium}.

There are also a lot of notes in the middle space of the four lines going across the page (the four lines are called the \emph{staff}).  We see the Do-clef symbol marking the top line, so the middle space is Sol.  That's the starting note.  It can be helpful to sing the solfege syllables when you are learning the tune.  The first part is: \textbf{sol mi fa re mi sol la}.  The little `b' symbol turns the note \textbf{ti} a semitone lower into \textbf{te}, so the next part is \textbf{la te la la sol fa sol sol}.

You probably want to hear the tune for yourself.  You can find recordings on the internet, either at kidschant.com or searching for Credo 1 on youtube.

The priest starts the first part: \emph{Credo in unum Deum}, then the choir takes up the next part: \emph{Patrem omnipotentem}\ldots.  We'll get on to the part where everyone joins in the singing in the next lesson.

Sometimes we take turns singing the Creed, changing voices every double bar line.  It helps make it a little easier on everyone that way.

\newpage

\section{Puzzle}

See if you can work out the Latin words from the clues below.

All the answers are in the words in the box.

\begin{Puzzle}{17}{15}
\Frame{12}{0}{5}{7}{Credo in unum Deum, Patrem omnipotentem, factorem caeli et terrae, visibilium omnium et invisibilium.}
|{} |{} |{} |{} |{} |{} |{} |{} |{} |{} |{} |{} |{} |{} |{} |[1]C |{} |.
|{} |{} |{} |{} |{} |{} |{} |[2]I |{} |{} |{} |{} |[3]D |{} |{} |R |{} |.
|{} |{} |{} |{} |{} |[4]O |M |N |I |P |[5]O |T |E |N |T |E |M |.
|{} |{} |{} |{} |{} |{} |{} |V |{} |{} |M |{} |U |{} |{} |D |{} |.
|{} |{} |{} |{} |{} |{} |{} |[6]I |[][r]N |U |N |U |M |{} |{} |O |{} |.
|{} |{} |{} |{} |{} |{} |{} |S |{} |{} |I |{} |{} |{} |{} |{} |{} |.
|{} |{} |[7]V |I |S |I |B |I |L |I |U |M |{} |{} |{} |{} |{} |.
|{} |{} |{} |{} |{} |{} |{} |B |{} |{} |M |{} |{} |{} |{} |{} |{} |.
|{} |{} |{} |{} |{} |{} |{} |I |{} |{} |{} |{} |{} |{} |{} |{} |{} |.
|{} |{} |{} |{} |{} |[8]T |{} |L |{} |{} |{} |{} |{} |{} |{} |{} |{} |.
|{} |{} |{} |[9]C |A |E |L |I |{} |{} |{} |{} |{} |{} |{} |{} |{} |.
|{} |{} |{} |{} |{} |R |{} |[10]U |N |U |M |{} |{} |{} |{} |{} |{} |.
|[11]F |A |C |T |O |R |E |M |{} |{} |{} |{} |{} |{} |{} |{} |{} |.
|{} |{} |{} |{} |{} |A |{} |{} |{} |{} |{} |{} |{} |{} |{} |{} |{} |.
|{} |[12]P |A |T |R |E |M |{} |{} |{} |{} |{} |{} |{} |{} |{} |{} |.
\end{Puzzle}

\begin{PuzzleClues}{\textbf{Across}}
\Clue{4}{}{almighty (obj) (12)} 
\Clue{6}{}{in one (2,4)} 
\Clue{7}{}{visible (10)} 
\Clue{9}{}{of heaven (5)} 
\Clue{10}{}{one (4)} 
\Clue{11}{}{maker (8)} 
\Clue{12}{}{Father (obj) (6)} 
\end{PuzzleClues}%
\begin{PuzzleClues}{\textbf{Down}}
\Clue{1}{}{I believe (5)} 
\Clue{2}{}{invisible (12)} 
\Clue{3}{}{God (obj) (4)} 
\Clue{5}{}{all things (6)} 
\Clue{8}{}{of earth (6)} 
\end{PuzzleClues}

\newpage

\section{Summary}

You have gone through 2 of the 18 sections of the longest prayer that we sing each Sunday at Mass!  Well done!

There are quite a few words to learn.  Some you might know already.  Some are easy to guess (like \emph{visibilium} and \emph{invisibilium}).

This Creed was written hundreds of years ago.  It is called the Nicene Creed, after the council of Nicea where the bishops gathered to fight heresy and make the truth clear.

In the Mass it forms the gateway between the Mass of the Catechumens (or Liturgy of the Word) and the Mass of the Faithful (or Liturgy of the Eucharist).  It was the Symbolon or password which distinguished initiated Christians from the rest.

Once you learn this, you will be carrying a precious key in your memory which has been passed down through the centuries.  In heaven you will be able to sing along with all the hundreds of saints who sang this very same music while they were on earth.

\PuzzleSolution 


\begin{Puzzle}{17}{15}
|{} |{} |{} |{} |{} |{} |{} |{} |{} |{} |{} |{} |{} |{} |{} |[1]C |{} |.
|{} |{} |{} |{} |{} |{} |{} |[2]I |{} |{} |{} |{} |[3]D |{} |{} |R |{} |.
|{} |{} |{} |{} |{} |[4]O |M |N |I |P |[5]O |T |E |N |T |E |M |.
|{} |{} |{} |{} |{} |{} |{} |V |{} |{} |M |{} |U |{} |{} |D |{} |.
|{} |{} |{} |{} |{} |{} |{} |[6]I |[][r]N |U |N |U |M |{} |{} |O |{} |.
|{} |{} |{} |{} |{} |{} |{} |S |{} |{} |I |{} |{} |{} |{} |{} |{} |.
|{} |{} |[7]V |I |S |I |B |I |L |I |U |M |{} |{} |{} |{} |{} |.
|{} |{} |{} |{} |{} |{} |{} |B |{} |{} |M |{} |{} |{} |{} |{} |{} |.
|{} |{} |{} |{} |{} |{} |{} |I |{} |{} |{} |{} |{} |{} |{} |{} |{} |.
|{} |{} |{} |{} |{} |[8]T |{} |L |{} |{} |{} |{} |{} |{} |{} |{} |{} |.
|{} |{} |{} |[9]C |A |E |L |I |{} |{} |{} |{} |{} |{} |{} |{} |{} |.
|{} |{} |{} |{} |{} |R |{} |[10]U |N |U |M |{} |{} |{} |{} |{} |{} |.
|[11]F |A |C |T |O |R |E |M |{} |{} |{} |{} |{} |{} |{} |{} |{} |.
|{} |{} |{} |{} |{} |A |{} |{} |{} |{} |{} |{} |{} |{} |{} |{} |{} |.
|{} |[12]P |A |T |R |E |M |{} |{} |{} |{} |{} |{} |{} |{} |{} |{} |.
\end{Puzzle}


\end{document}
